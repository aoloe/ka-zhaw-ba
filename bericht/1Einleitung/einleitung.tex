\chapter{Einleitung}
(koenigma)
In der heutigen Zeit gibt es viele interessante Gadgets, die unterschiedlichste Daten liefern. Seien das Pulsmesser, Heizungsregler oder das Multimediasystem zu Hause, diese Technologien lassen sich auch für den Fahrradfahrer nutzen. Es gibt bereits sogenannte Fahrradcomputer, welche die Geschwindigkeit messen und über ein separates Display ausgeben, jedoch werden die meisten mit einer Batterie betrieben, deren Laufzeit begrenzt ist. Mit der Möglichkeit des Energy Harvesting wird die Batterie und deren begrenzte Laufzeit gänzlich ersetzt. Bluetooth Low Energy kann Daten mit sehr wenig Energie übertragen, damit können die Daten, wie Geschwindigkeit oder Höhenmeter, an ein Android-Endgerät übermittelt werden.

(bachlkat)
Es gibt viele interessante Gadgets mit interessanten Daten für den Alltag: Pulsmesser, Heizungsverbrauch, ... (Sensoren). So auch für Velofahrer. Hier gibt es die Erweiterung des traditionellen Tachometers, der die berechneten Daten auf das Handy ausgibt. 
Die meisten dieser Gadget benötigen eine Batterie. Hier gibt es den neuen Ansatz, die Energie aus der Umgebung zu ernten, auf englisch energy harvesting. In dieser Bachelorarbeit wird ein Tachometer für das Velo entwickelt, der keine Batterie gebraucht. Die Daten dieses Tachometers werden an ein Android-Handy gesendet und dort angezeigt.
Fragen:
Einleitung: Wort Tachometer korrekt ?
Englische Worte: immer kursiv und klein ?

\section{Ausgangslage}
Tachometer auf Handy.
Fahrradcomputer.
- Halterung für das Smartphone
- Speisung Tacho
- Internetrecherche PA \cite{PA_bicycle} 
   "bicycle with a bicyle-mounted energy collector"
   "bicycle electrical generator hub"
   "Electric generator for bicylce"

Fragen:
Welche Aspekte auflisten ?  
Bilder ?

\section{Aufgabenstellung}



\section{Zielsetzung}





\section{Übersicht der Arbeit}
