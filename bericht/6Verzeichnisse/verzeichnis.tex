\chapter{Verzeichnisse}




\renewcommand{\bibsection}{\section{\refname}}  % add a number to the bibl section
\makeatletter
%\renewcommand*\bib@heading{ \section{\refname}}
\makeatother


\bibliography{BibTex/references}



\section{Glossar und Abkürzungen}\label{glossar}

\textbf{Clock Domain}\\
Ein Bereich der Hardware, der mit demselben Takt läuft.

\textbf{Power Domain}\\
Basiert auf der Fähigkeit eines Prozessor Speisungsgebiete zur Verfügung zu stellen. Der Prozessor teilt seine Funktionalitäten in Gebiete ein, die separat ein- und ausgeschalten werden können.

\textbf{MPP}\\
Maximum Power Point (MPP) bezeichnet in einer Leistungskurve den höchsten Punkt, also das Leistungsmaxiumum.

\textbf{MPPT}\\
Versucht ein System, einen Input stets auf das Leistungsmaximum zu regeln, spricht man von Maximum Power Point Tracking. Tracking steht für Einfangen.

\textbf{MPPT-Ratio}\\
Bezeichnet die Auswertung des MPP auf Spannungsachse. Liegt das Leistungmaximum beim Kurzschluss, so ist die MPPT-Ratio bei 0 \%, liegt sie bei Leerlauf, dann liegt die MPPT-Ratio bei 100 \%. Üblicherweise liegt die MPPT-Ratio dazwischen.

\textbf{State Machine}
Heisst korrekt Finite State Machine und bezeichnet eine Konzept, bei dem aufgrund einer Kombination von Eingangssignalen, sich das System in einem bestimmten Zustand befindet. In jedem Zustand sind nur gewisse Inputs zulässig, ansonsten verbleibt das System in diesem Zustand. Folgt ein korrekter Input, wechselt das System in den entsprechenden Zustand. 



\textbf{UML}\\
Die Unified Modeling Language (UML) ist ein Quasistandard, wie Prozesse abgebildet werden können. Die Sprache definiert Formen, aufgrund deren man weiss, ob es sich um eine Initialiserung, eine Entscheidung oder um eine Verarbeitung, etc. handelt.


\section{Abbildungsverzeichnis}


\section{Tabellenverzeichnis}

