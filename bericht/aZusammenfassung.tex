\chapter{Zusammenfassung}

Internet of Things, also die Möglichkeit, Daten unter Geräten auszutauschen, soll für einen Fahrradfahrer nutzbar gemacht werden. Das  mobile Gerät soll durch Energy Harvesting gespiesen werden und bei einer Geschwindigkeit von 10 km/h Sensordaten mit Bluetooth Low Energy (BLE) senden. Die Arbeit baut auf einer Machbarkeitsstudie auf, die das Management der gewonnenen Energie mit dem Chip EM8500 löst. Als Sendemodul wird das SensorTag von Texas Instruments der Serie Simple Link benutzt. Dieses Board beinhaltet einen Wireless MCU und den Low Power Cortex M3. 

Aufgabe der Arbeit sind das Entwickeln einer minuaturisierten Leiterplatte, die nicht grösser als das verwendete TI-SensorTag ist, das Einstellen von Speicherelementen und von Schwellwerten für die Energiegewinnung und das Power optimieren des Codes. Als Produkt steht neben der Hardware eine benutzerfreundliche Android Applikation zur Verfügung. Diese beinhaltet Einstellungen der Sensoren und einen ansprechenden Tachometer, der die Fahrgeschwindigkeit anzeigt. 

Anfangs wird der Aufbau der Machbarkeitsstudie in Betrieb genommen. Die bestehende Version sendet Geschwindigkeit ab 45 km/h und basiert auf einem fliegenden Aufbau.
Nach der verbesserung der Harvesterschaltung, sodass bei 10 km/h rund 20 $\mu$W zur Verfügung stehen, wird der Print designt. Das Energiemanagment im EM8500-Chip und die Firmware des TI-SensorTags werden komplett neu geschrieben. Die Schwellwerte beim Energy Management basieren auf dem MPP des Harvesters und dem Ziel, konstant BLE-Daten bei 10 km/h zu senden. Bei der Firmware das TI-SensorTags werden über Sleep-Funktionen dem System genügend Zeit zum wieder Aufladen gegeben. 

Der Prototyp ist eine konfigurierbare BLE-Applikation, die bei 10 km/h jede zweite Minute Geschwindigkeits-, Druck-, Temperatur- und Luftfeuchtigkeitsdaten erhält. Bei 20 km/h werden die Daten nach 20 s und bei über 45 km/h konstant aktualisiert.