\chapter{Diskussion}

Am Ende der Arbeit steht ein lauffähiger Prototyp eines Bicycle Computers zur Verfügung. Die Leistungsgewinnung ist um xxx \todo{Faktor} optimiert, der Aufbau minimiert und auf eine Leiterplatte gebracht, das Energy Management auf die Fahrgeschwindigkeit von 10 km/h optimiert sowie neu werden auch Sensordaten an die benutzerfreundliche Android-Applikation gesendet. %Die Resultate sind grösstenteils mathematisch und technisch erklärbar, nur die Funktionsweise des EM8500-Chips, sowie des TI-SensorTags war teilweise nicht erwartungsgemäss, jedoch befindet sich der EM8500-Chip noch in der Entwicklung und das TI-SensorTag wurde nicht mit dem vorgesehenen RTOS verwendet.

Der Prototyp ist so entwickelt, dass zukünftige Teams den Code schnell verstehen, die Leiterplatte leicht aufteilen und weiterentwickeln können und die modulare Android-Applikation ist nach belieben ausbaubar.

Aus der zu Beginn definierten Aufgabenstellung sind alle Minimalanforderungen erreicht. Folgende optionale Aufgaben sind ebenfalls eingebettet worden: \todo{optionale Aufagaben auflisten}.

Als mögliche Weiterentwicklungen stehen insbesondere Energieverbrauchsoptimierungen an erster Stelle. Das Ziel ist, dass bis zur Präsentation des Bicycle Computer an der Nacht der Technik, die Sensoren mit weniger Energie ausgelesen werden, sodass die Versorgungsspannung bei Geschwindigkeiten unter 40 km/h nicht abstellt. Interessant ist die Auswirkung des Connected Modes bei höherer Geschwindigkeit. Bei tiefer ist der Verbindungsaufbau nicht realistisch. Doch bei höheren Geschwindigkeiten könnte über den Connected Mode sicher Daten versendet werden.

Die entwickelte Leiterplatte kann für Schülerinnen und Schüler zum Experimentieren mit BLE verwendet werden. Bewusst wurden viele Testpunkte und ein Stecker für das Abgreifen der Signale implementiert. Auch bestehen Löcher für Füsse \todo{wie heissen diese} und die Kondensatoren können flexibel angelötet werden \todo{besser formulieren}. Das TI-SensorTag zusammen mit dem EM8500 eignen sich gut zum Kennenlernen des Energy Managements. Das Erstellen der Leiterplatte hat sicher gelohnt.

Für die Aufführung des fertigen Produkts ist ein Gehäuse in Planung. Dieses wird an der Verstrebung Richtung Sattel. Die Distanz zum Rad, kann flexibel eingestellt werden, sodass der Prototyp sich nicht auf ein Fahrradmodell limitiert. Zum Endprodukt zählt auch eine professionellen Montage der Doppelmagnete an den Speichen. Als Letztes wird der Hohlraum zwischen der Leiterplatte und dem TI-SensorTag durch zwei Verstrebungen verstärkt, damit das Gerät Bodenschläge aushält. Das Endprodukt ist somit ein voll anwendungsfähiger Bicycle Computer. 

Über das Produkt sind wir überaus Stolz. Dies nicht zu Letzt, da die Umsetzung nicht ganz einfach war. Der EM8500-Chip ist ein neues Produkt und verhält sich teilweise nicht stabil. Konkret musst Manuel König acht EM8500-Chips auf die Leiterplatten anlöten, weil immer wieder einer ausstieg. Das heisst, entweder konnte die Kommunikation über SPI nicht mehr statt finden, Slave-Address-Error, oder V_SUP startete trotz grosser Energie am Eingang nicht. Da der Chip nicht zuverlässig funktionierte, war es in der Entwicklung schwierig zu unterscheiden, ob in der Harversterschaltung ein Fehler auftrat oder der Chip sich aufhängte. Zu oft suchten wir den Fehler in der Hardware und nicht im Chip. Die zweite Herausforderung ist die Benutzung des TI-SensorTags ohne RTOs. Diese rudimentäre Programmierung machte enorm Spass. Und mit Hilfe des Know-Hows am InES konnten die Probleme längerfristig gelöst werden. Da es Pionierarbeit ist, ging die Entwicklung des Codes nicht so schnell wie gewünscht vorwärts. Sinnvoll ist z.B. bei mehr Energie per SPI das Status Register des EM8500 auszulesen. In diesen 8 Bits steht der Zustand des Energyzustands der Speicher und welche Schwellwerte überschritten sind. 
(Diese elegante Methode ist als Funktion abgelegt und der Energiezustand des Systems könnte feiner abgestimmt werden, als durch eine Energiezustandsdefinition über die aktuelle Geschwindigkeit.)

Abschliessend möchten wir sagen, dass die Unterstützung bei der Bachelorarbeit sowohl von Prof. Dr. Meli wie auch von Research Assistent Dario Dündar sehr gut war. Die Problem, die auf uns zukamen, waren viel grösser als erwartet. Da wir es super im Team hatten, wir ergänzen uns auf gute Art und Weise, ging uns die Freude trotzt unerklärlichem, technischen Verhalten nie aus. Durch diese Arbeit haben wir sehr viel gelernt und freuen uns, zukünftig als Ingenieurin und als Ingenieur zu arbeiten.





\subsection{Reflexion von Herr Manuel König}
Die Arbeit war sehr interessant, jedoch gab es mehrer Punkte die erarbeitet werden mussten, welche im vorherein nicht absehbar waren. Anfangs war die Meinung, dass es sich hauptsächlich um eine Softwarearbeit handelt, welche ein wenig Hardwareprobleme beinhaltet. Die Optimiertung der Hardware war sehr komplex und ich hätte gerne noch mehr Zeit für die Optimierung aufgewendet, jedoch hätten wir dann die Ziele nicht mehr erreicht. Die Arbeit im Team war sehr lösungsorientiert, es wurde nie ein Schuldiger für ein Problem gesucht, sondern immer Lösungsvorschläge generiert. Die verschiedenen Arbeitsweisen waren interessant zu beobachten, jedoch war es nie ein Problem dass unterschiedlich gearbeitete wurde. Die Teamarbeit war sehr angenehm und ich würde jederzeit wieder mit Frau Katrin Bächli zusammen arbeiten.




