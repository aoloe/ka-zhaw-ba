\chapter{Abstract}

The exchange of data between devices - commonly referred to as
\glqq Internet of things\grqq - should be made accessible to cyclists. The
mobile device should start to work at a cycling speed of 10km/h,
transmitting data through the Bluetooth Low Energy (BLE) protocol. The
task is based on a feasibility study which handles the energy
management through the chip EM8500. A \glqq SensorTag\grqq - available on Texas
Instruments Simple Link devices - is responsible for sending the data.
This board contains a wireless MCU and a low power Cortex M3.

The goal of the task was the development of a miniaturised board, which
should not be bigger than the utilized TI-SensorTag while adjusting the
energy storage, energy management and the power consumption of the
code. The finished product contains the hardware and a user friendly
Android application. The \glqq app\grqq allows adjusting the sensor readouts and
displays a tachometer and a speedometer showing the current speed, which is based on the radius of the wheel).

At the start, a feasibility study was conducted. The feasibility study
version showed promise for enough energy being produced at speeds
over 45km/h.

After improving the energy harvest we were able to generate 20
$\mu$W at 10km/h. The energy management of the EM8500-Chip and the
firmware of the TI-SensorTag were then completely rewritten. The
threshold of the energy management is based on the MPP of the harvester
and the goal is to send constant BLE-data at 10km/h. The firmware of
the TI-SensorTag uses sleep functions to reduce the energy
requirements.

The prototype is a configurable BLE-application, which sends speed,
pressure, temperature and air moisture data in two minute intervals at
10km/h. From 20km/h onward, the data is sent every 20 seconds and over
45km/h the data is sent continuously.
